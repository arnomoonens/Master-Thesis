\chapter{Conclusion}

We presented an algorithm suitable for learning in parallel or sequentially on a set of source tasks.
These task share a knowledge base, but they also have their proper sparse representation.
The learned knowledge can then be transferred to the target task with the goal of having an increased performance over an algorithm that does not use prior knowledge.\\

\todo{Something about applying knowledge immediately in results section}
In our experiments, we found that source tasks learned in parallel learned faster than when they were learned sequentially.
In the parallel version, changes to the shared knowledge base are applied immediately instead of summing them and applying the changes after all source tasks have been evaluated.\\

We also discussed to use of feature extraction applied to the input, which is in this case the state of an environment.
For the cart-pole environment, no feature extraction was necessary and even slowed down learning.
However, environments with a high-dimensional input space generally require feature extraction in order to obtain the relevant aspects of the input.\\

Our algorithm has better performance on the target task than when just using the \textit{REINFORCE} algorithm on it.
Our algorithm learns faster and is able to receive higher rewards.
The performance is even better when we transfer the sparse representation from a randomly chosen source task to the target task.
The algorithm then only needs to tune the sparse representation for it to work on its own task.\\

To see if multiple source tasks are really necessary, we compared our algorithm with the \textit{REINFORCE} algorithm where it learns on a single source task and transfers all its knowledge to the target task.
Although the asymptotic performance was similar, our algorithm learned better on the source tasks and has a higher jumpstart performance.\\

We can conclude that it is beneficial to learn on multiple source tasks in parallel and transfer knowledge learned on these tasks to the target task.
