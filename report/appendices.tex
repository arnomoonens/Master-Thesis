\begin{appendices}
\chapter{Experiment details}
\section{Data collection} % (fold)
\label{sec:data_collection}
For each agent, transitions on which to learn were collected in the same way. In each iteration, $5000$ transitions have to be collected. To do this, we keep executing actions and saving the transitions until this threshold is reached. When an end state is reached before reaching this threshold, we reset the environment and keep executing actions.
% section data_collection (end)
\section{Artificial neural network parameters} % (fold)
\label{sec:artificial_neural_network_parameters}
Weights are initialized using values drawn from a truncated normal distribution with mean $0$ and standard deviation $0.02$.
This means that we draw values from a normal distribution with the same mean and standard deviation, except that only values maximally 2 standard deviations away from the mean can be drawn.\\
Our transfer learning algorithm always used an $L_1$ loss for the sparse representations, with $\lambda = 0.05$.
The sparse representation has a size of $10 \times n_a$, where $n_a$ is the number of actions for the environment. The knowledge base has the size $d^{l_1}  \times 10$, where $d^{l_1}$ is the number of features in the state or after feature extraction.
To update weights, I used \textit{RMSProp}, described in Section~\ref{ssub:different_weight_updates}.
The following hyperparameters were always used for both the sequential and the parallel version of the algorithm and for \textit{REINFORCE}:
\begin{itemize}
    \item $\epsilon = 10^{-9}$
    \item $\alpha = 0.05$
    \item $\gamma = 0.9$
\end{itemize}

\section{Environment parameters} % (fold)
\label{sec:environment_parameters}
Here we describe the range of possible values for each parameter of the \textit{cart-pole} and \textit{acrobot} environments.
To generate an environment, each parameters is drawn from a uniform distribution with possible values in the defined range.
For the \textit{cart-pole} environment, we have:
\begin{table}[H]
    \centering
    \begin{tabular}{l|cc}
    \hline
    \textbf{Parameter name} & \textbf{minimum} & \textbf{maximum} \\
    \hline
        Pole length & $0.01$ & $5.0$ \\
        Pole mass & $0.01$ & $5.0$ \\
        Cart mass & $0.01$ & $5.0$ \\
    \hline
    \end{tabular}
\end{table}
For the \textit{acrobot} environment, we have:
\begin{table}[H]
    \centering
    \begin{tabular}{l|cc}
    \hline
    \textbf{Parameter name} & \textbf{minimum} & \textbf{maximum} \\
    \hline
        Length of link 1 & $0.2$ & $2.0$ \\
        Length of link 2 & $0.2$ & $2.0$ \\
        Mass of link 1 & $0.2$ & $2.0$ \\
        Mass of link 2 & $0.2$ & $2.0$ \\
    \hline
    \end{tabular}
\end{table}

% section environment_parameters (end)
\end{appendices}
