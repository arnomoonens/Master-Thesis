\documentclass{beamer}
\usepackage{hyperref}

%\usetheme{vub}
\usetheme[coloredtitles]{vub}
%\usetheme[showsection]{vub}

\title{Knowledge transfer in deep reinforcement learning}
\subtitle{Master thesis}
\author{Arno Moonens}
\date{June 30, 2017}

% To center the contents of a frame vertically
\makeatletter
\define@key{beamerframe}{c}[true]{% centered
  \beamer@frametopskip=0pt plus 1fill\relax%
  \beamer@framebottomskip=0pt plus 1fill\relax%
  \beamer@frametopskipautobreak=0pt plus .4\paperheight\relax%
  \beamer@framebottomskipautobreak=0pt plus .6\paperheight\relax%
  \def\beamer@initfirstlineunskip{}%
}
\makeatother

% To change footnote size
\let\oldfootnotesize\footnotesize
\renewcommand*{\footnotesize}{\oldfootnotesize\tiny}

\setbeamertemplate{caption}{\raggedright\insertcaption\par}

\AtBeginPart{\frame{\partpage}}
% \AtBeginSection{\frame{\sectionpage}}
% \AtBeginSubsection{\frame{\subsectionpage}}

\graphicspath{ {../images/} }

\begin{document}
\frame{\titlepage}

\begin{frame}{Outline}
    \vskip0pt plus 2fill
  {\color{vubbleu}\large Background}
  \tableofcontents[part=1]

  {\color{vubbleu}\large Experiments}
  \tableofcontents[part=2]
\end{frame}

\part{Background}

\section{Artificial neural networks}
\begin{frame}[fragile]\frametitle{Artificial neural networks}
\framesubtitle{Architecture}
\vskip0pt plus 3fill
\begin{columns}
\begin{column}{0.5\textwidth}
\includegraphics[width=\linewidth]{neuron.png}
\end{column}
\vrule
\begin{column}{0.5\textwidth}
\includegraphics[width=\linewidth]{ann.png}
\end{column}
\end{columns}
\end{frame}

\section{Reinforcement learning}
\begin{frame}[fragile]\frametitle{Reinforcement learning}
\vskip0pt plus 1fill
\begin{center}
    \includegraphics[width=.7\linewidth]{reinforcementlearning.png}
    \footnote{Source: \url{https://www.analyticsvidhya.com/blog/2017/01/introduction-to-reinforcement-learning-implementation/}}
\end{center}
\begin{itemize}
    \item Goal: maximize a reward
    \begin{itemize}
        \item By changing its policy $\pi$
    \end{itemize}
    \item Policy gradient:
    \begin{itemize}
        \item Policy is parametrized: $\pi_\theta(s,a)$
        \item Input: states
        \item Output: action (probabilities)
    \end{itemize}
\end{itemize}
\end{frame}

\section{Deep reinforcement learning}
\begin{frame}[fragile]{Deep reinforcement learning}
\framesubtitle{A3C architecture}
\vskip0pt plus 10fill
\begin{center}
    \includegraphics[width=.7\linewidth]{A3Carchitecture}
\end{center}
\end{frame}

\section{Transfer learning}
\begin{frame}[fragile]{Transfer learning}
% \framesubtitle{A3C architecture}
\vskip0pt plus 10fill
\begin{center}
    % \includegraphics[width=.7\linewidth]{A3Carchitecture}
\end{center}
\end{frame}

\section{Transfer learning}

\part{Experiments}
\section{Proposed algorithm}
\begin{frame}[fragile]{Proposed algorithm}
\framesubtitle{Architecture}
\vskip0pt plus 1fill
\begin{center}
    \includegraphics[width=\linewidth]{knowledge_transfer.png}
\end{center}
\end{frame}

\section{Experimental setup}
\begin{frame}[fragile]{Experimental setup}
\framesubtitle{Environments}
\begin{figure}[htb]
    \begin{minipage}{0.55\textwidth}
            \includegraphics[width=\linewidth]{cartpole.png}
            \caption{Cart-Pole}
    \end{minipage}\hfill
    \begin{minipage}{0.45\textwidth}
            \includegraphics[width=\linewidth]{acrobot.png}
            \caption{Acrobot}
    \end{minipage}
\end{figure}

\end{frame}

\section{Results}
\frame{\sectionpage}
\begin{frame}[fragile]{Parallel and sequential knowledge transfer}
\framesubtitle{Cart-pole}
\vskip0pt plus 3fill
\begin{center}
    \includegraphics[width=.8\linewidth]{results/CartPole/kt_akt/reward_source-target_5tasks.png}
\end{center}
\end{frame}

\begin{frame}[fragile]{Feature extraction}
\framesubtitle{Cart-pole}
\vskip0pt plus 3fill
\begin{center}
    \includegraphics[width=.8\linewidth]{results/CartPole/feature_extraction.png}
\end{center}
\end{frame}

\begin{frame}[fragile]{Different amount of source tasks}
\framesubtitle{Cart-pole}
\vskip0pt plus 3fill
\begin{center}
    \includegraphics[width=.8\linewidth]{results/CartPole/no_sparse_transfer/reward_target_re-akt5-akt10.png}
\end{center}
\end{frame}

\begin{frame}[fragile]{Different amount of source tasks}
\framesubtitle{Acrobot}
\vskip0pt plus 3fill
\begin{center}
    \includegraphics[width=.8\linewidth]{results/Acrobot/no_sparse_transfer/reward_target_re-akt5-akt10.png}
\end{center}
\end{frame}

\begin{frame}[fragile]{Transfer of sparse representation}
\framesubtitle{Cart-pole}
\vskip0pt plus 3fill
\begin{center}
    \includegraphics[width=.8\linewidth]{results/CartPole/sparse_transfer/reward_target_re-akt5-akt10.png}
\end{center}
\end{frame}

\begin{frame}[fragile]{Transfer of sparse representation}
\framesubtitle{Acrobot}
\vskip0pt plus 3fill
\begin{center}
    \includegraphics[width=.8\linewidth]{results/Acrobot/sparse_transfer/reward_target_re-akt5-akt10.png}
\end{center}
\end{frame}

\begin{frame}[fragile]{REINFORCE using a source and target task}
\framesubtitle{Cart-pole}
\vskip0pt plus 3fill
\begin{center}
    \includegraphics[width=.8\linewidth]{results/CartPole/reinforce_2tasks.png}
\end{center}
\end{frame}

\begin{frame}[fragile]{REINFORCE using a source and target task}
\framesubtitle{Acrobot}
\vskip0pt plus 3fill
\begin{center}
    \includegraphics[width=.8\linewidth]{results/Acrobot/reinforce_2tasks.png}
\end{center}
\end{frame}

\begin{frame}[c]{The end}
\begin{center}
    \color{vubbleu} \LARGE\vubfont Thank you for listening!
\end{center}
\end{frame}

\end{document}
